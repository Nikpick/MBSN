%!TEX root = ../../../../root.tex

Un'istanza della classe \textit{Type} rappresenta un data type disponibile nel modello. Le componenti di rilevanza per questa classe sono:
\begin{itemize}
	\item nome: stabilisce il nome del data type.
\end{itemize}
La classe \textit{Type} è una classe astratta, utilizzata per la rappresentazione delle classi più specifiche \hyperref[sec:modelstranslator:anlysis:model_analysis:type:basetype]{\textit{BaseType}} e \hyperref[sec:modelstranslator:analysis:model_analysis:module]{\textit{Module}}. \\
La classe \textit{Module} rappresenta anche un data type poiché in un modello è possibile definire variabili che rappresentano altri modelli; questo è utile per rappresentare le connessione tra i vari modelli.

\InputSection[sec:modelstranslator:anlysis:model_analysis:type:basetype]{Classe BaseType}{BaseType/content.tex}